\section{$\mathbb{K}$ keretrendszer}

Diplomamunkám témája formális szemantikát definiálni az Erlang egy résznyelvéhez nem a megszokott módszerekkel, mint például denotációs vagy operációs szemantika, hanem a $\mathbb{K}$ keretrendszer segítségével. Ez a keretrendszer képes a szemantikai szabályok alapján egy értelmezőt készíteni, amivel programjaink futtathatóak, így ténylegesen megtekinthetjük az általunk definiált szemantikánk működését. Ezen felül még sok más érdekes funkcióval is rendelkezik. A keretrendszer fejlesztését Grigore Rosu kezdte 2003-ban \cite{KLecture}. Jelenleg az amerikai Illinois Urbana-Champaign Egyetem és a román Alexandru Ioan Cuza Egyetem közös projektje.

Az egyik ok, amiért ez a projekt létre jött, az eszközök hiánya formális szemantika definiálására. Ezenfelül ha definiálunk is valamilyen más módszerrel, nem lehet egyszerűen értelmezőt létrehozni, nem lehet helyesség bizonyítani programjainkat közvetlenül a szemantika alapján, a létrehozott modellt nem tudjuk ellenőrizni, hogy megfelel-e az adott specifikációnak és még sok minden felsorolható. Ha van egy formális nyelvdefiníciónk, akkor elvben az előbb felsoroltak megvalósíthatóak közvetlenül a nyelvdefiníciót használva, és nem kellene teljes mértékben támaszkodnunk a sok esetben ad-hoc módon implementált fordítókra. Ezt a célt tűzték ki a keretrendszer fejlesztői. A formális nyelv definiálására létrehozott módszer nem rendelkezik a már ismertek hátrányaival, mint például a modularitási vagy verifikálási nehézségek.

\subsection{Szemantika definiálása}

A keretrendszerrel teljes formális nyelvdefiníció készíthető, tehát nem csak a szemantika, hanem a szintaxis megadása is kötelező mégpedig Backus–Naur-formában (BNF). Ezekhez különböző attribútumokat társíthatunk. Közülük a legfontosabb a \textit{strict}, amely a kiértékelési stratégiát határozza meg. Az ilyen fajta megadási módszer kézenfekvő, hisz szemantikaszabályok esetén már csak a kiértékelt értékekkel kell foglalkozni.

\begin{greyBox}

\begin{syntaxBlock}{\nonTerminal{\sort{Exp}}}
  \syntax
    {\nonTerminal{\sort{Exp}} = \nonTerminal{\sort{Exp}}}
    {\kattribute{strict(2)}}
\end{syntaxBlock}

\end{greyBox}

A fenti példában az értékadás operátor szintaxisa látható BNF jelöléssel, és a hozzátartozó kiértékelési stratégiával. Az operátor fajtája \textit{Exp} és a két operandusé is ugyanaz, vagyis az értékadás mind két oldalán egy kifejezés áll, és az értékadás önmagában is egy kifejezés. A \textit{strict} attribútum után zárójelben egy kettes van, így a meghatározott kiértékelési stratégia: csak a második operandust értékeljük ki, így az értékadó operátorhoz tartozó szemantikai szabály csak ezután alkalmazható.

\begin{greyBox}

\begin{kblock}

\kconfig{
\kall{yellow}{T}{
  \kall{\K_COLOR}{k}{
    \variable[Pgm]{\$PGM}{user}
  }
  \mathrel{}
  \kall{\DEFS_COLOR}{defs}{
    \dotCt{Map}
  }
  \mathrel{}
  \kall{\ENV_COLOR}{env}{
  	\dotCt{Map}
  }
}}

\end{kblock}

\end{greyBox}

A keretrendszerben a definiált nyelvhez tartozik egy konfiguráció, ami egymásba ágyazott cellákból áll a sorrend figyelembe vétele nélkül. Ezek tartalmazzák az összes információt, amire a programnak, illetve nekünk elemzés szempontjából szükségünk van. A konfiguráció gyakorlatilag az állapotot tartalmazza, illetve egy speciális \textit{k} cellát, mely a számításokat tartalmazza, ami gyakorlatilag az absztrakt szintaxis fa kiegészítve egy lista struktúrával ami a $\sim>$ jelet használja mint szeparátor. Elemei speciális \textit{K} fajtájúak. A listára bontást a keretrendszer automatikusan elvégzi a \textit{syntax} kulcsszó alapján. Ennek eredménye hogy a szemantikaszabályok egy egységes \textit{K} fajtákat (\textit{term}) tartalmazó számítási folyamra tudnak illeszkedni.

%Eddig OK

\begin{greyBox}

\begin{kblock}

\krule{
    \kprefix{\K_COLOR}{k}{
      \reduce
        {\variable[]{X}{} \mathrel{=} \variable[]{V}{}}
        {\variable[]{V}{}}
    }
    \mathrel{}
    \kmiddle{\ENV_COLOR}{env}{
      \variable[]{X}{} \mapsto
      \reduce
        {\terminal{\_}}
        {\variable[]{V}{}}
    }
  }
  {}{}{}{}
%
<k> X = V => V ...</k>
\newline
<env>... X |-> (\_ => V) ...</env>
         
\end{kblock}         
         
\end{greyBox}

Az értékadó operátornak a szemantika szabályát mutatja be a fent látható példa. Felül az olvashatóbb, prezentációra szánt változat, melyet generáltatni lehet a keretrendszerrel, illetve alul a ténylegesen írott forma. Látható hogy két cellát a \textit{k}-t és a az \textit{env}-et tartalmazza. Ha a kiértékelési stratégia végbement, akkor a fenti szabály alkalmazható. Ezek a szabályok a redukciós szabályok, annak ellenére, ha nem is csökkentik a konfigurációt. Az átmenet a \textit{k} cellában az értéket tartja meg, és ezalatt az \textit{env} cellában az \textit{X} K változóban lévő programbeli változóhoz hozzárendeli az értéket. Az átmenetben látható a \textit{\_} joker karakter, ami mindenre tud illeszkedni, tehát akármi is volt ehhez a változóhoz hozzárendelve azzal nem foglalkozunk többet. A szabály csak két cellát tartalmaz, azokat amik az értékadó operátor redukciós szabályában szerepet játszanak, a többivel nem foglalkozik. Ennek előnye hogy a konfiguráció könnyedén bővíthető úgy, hogy nem kell a már meglévő szabályhalmazunkon módosítani. Az alsó résznél látható, hogy a \textit{k} cellában jobb oldalon három pont látható, illetve az \textit{env} cellában mind a két szélén. Ennek a jelentés az, hogy a termek, amikez a cellák tartalmaznak, az a cella elejére, végére vagy lényegtelen, hogy hova illeszkedjen. A szabályokhoz tartozhatnak különböző feltételek, amelyeknek teljesülniük kell, hogy az illeszkedés végbemehessen.

Most már visszatérhetünk pontosan mit is csinál eleméletben a \textit{strict} attribútum. Példaként vegyük megint az értékadás operátort. A hozzátartozó attribútum az alábbi két szabályt generálja.

\begin{greyBox}

\begin{kblock}

\krule{
    \reduce
      {\variable[]{X}{} \mathrel{=} \variable[]{A}{};}
      {\variable[]{A}{} \kra \variable[]{X}{} \mathrel{=} [];}
  }
  {}{}{}{}
%
\krule{
    \reduce
      {\variable[]{A}{} \kra \variable[]{X}{} \mathrel{=} [];}
      {\variable[]{X}{} \mathrel{=} \variable[]{A}{};}
  }
  {}{}{}{}
         
\end{kblock}
         
\end{greyBox}

Az első szabály kiveszi az értékadó operátor kontextusából a második operandust, és berakja a folyam elejére egy új termként. A második pedig a kivett operandust visszarakja az értékadó operátor környezetébe. A kémiai absztrakt gép alapján ezeket rendre fűtő és hűtő szabályoknak nevezzük. \textit{strict} attribútum esetén a sorrend nem számit.

\begin{greyBox}

\begin{kblock}

\krule{
    \reduce
      {\variable[]{A1}{} \mathrel{+} \variable[]{A2}{}}
      {\variable[]{A1}{} \kra [] \mathrel{+} \variable[]{A2}{}}
  }
  {}{}{}{}
%
\krule{
    \reduce
      {\variable[]{A1}{} \kra [] \mathrel{+} \variable[]{A2}{}}
      {\variable[]{A1}{} \mathrel{+} \variable[]{A2}{}}
  }
  {}{}{}{}
%
\krule{
    \reduce
      {\variable[]{A1}{} \mathrel{+} \variable[]{A2}{}}
      {\variable[]{A2}{} \kra \variable[]{A1}{} \mathrel{+} []}
  }
  {}{}{}{}
%
\krule{
    \reduce
      {\variable[]{A2}{} \kra \variable[]{A1}{} \mathrel{+} []}
      {\variable[]{A1}{} \mathrel{+} \variable[]{A2}{}}
  }
  {}{}{}{}
         
\end{kblock}         
         
\end{greyBox}

Itt látható szabályok abban az esetben generálodnak, ha az összeadás operátor kiértékelési stratégiája olyan, hogy mind a két operandust ki kell értékelni szabály alkalmazása előtt. Könnyen látható, hogy ezek a szabályok könnyen előidézhetnek nem determinisztikus futást. Ugyan ezek a szabályok generálodnak a \textit{seqstrict} attribútum esetén is, de itt már számít a sorrend. Tehát az \textit{A2}-re a egészen addig nem alkalmazza a fűtő/hűtő szabályokat, amíg az A1-re nem alkalmazta ezeket. Viszont a szabályok felcserélhetőek, aminek következménye, hogy a $\mathbb{K}$-ban definiált nyelv nem lesz futtatható, mivel előfordulhat hogy nem terminál. Ennek érdekében lett bevezetve a \textit{KResult} fajta és a hozzá tartozó \textit{isKResult} szemantikus függvény, amely eldönti egy termről, hogy \textit{KResult} fajta vagy sem. Ezzel a felcserélhetőséget úgy akadályozza meg, hogy a teljes generált szabályok értékadó operátor esetén az alábbiak:

\begin{greyBox}

\begin{kblock}

\krule{
    \reduce
      {\variable[]{X}{} \mathrel{=} \variable[]{A}{};}
      {\variable[]{A}{} \kra \variable[]{X}{} \mathrel{=} [];}
  }
  {\mathrel{\neg_{\scriptstyle\it Bool}} \terminal{isKResult}(\variable[]{A}{})}{}{}{}
%
\krule{
    \reduce
      {\variable[]{A}{} \kra \variable[]{X}{} \mathrel{=} [];}
      {\variable[]{X}{} \mathrel{=} \variable[]{A}{};}
  }
  {\terminal{isKResult}(\variable[]{A}{})}
  {}{}{}
         
\end{kblock}         
         
\end{greyBox}

Ilyenkor minden esetben először a fűtő szabályt alkalmazza, és egészen addig, amíg a kontextusból kiemelt term nem \textit{KResult} fajtájú, egészen addig nem tudja alkalmazni rá a hűtő szabályt.

\subsection{További területei}

Az előző alfejezetben láthattuk, hogyan definiálhatunk egy formális nyelvet a $\mathbb{K}$ keretrendszerrel. A kész definíciónkat a \textit{kompile} paranccsal fordíthatjuk le, és a \textit{krun} paranccsal futtathatjuk a generált értelmezőt programon. Alapértelmezett beállításokkal az előbb említett működést kapjuk, viszont konfigurálhatjuk hogy az összes lehetséges működést megkapjuk. Ez főleg konkurrens programoknál fordulhat elő. Tipikusan jó példa erre, mikor két szál egy globális változót akar módosítani. Az eredmény mindig függeni fog attól, hogy a gép hogyan ossza ki az órajelet a különböző szálaknak.

A keretrendszer mögött három backend is állhat. Az elsőnek implementált verzióból let a Maude nevű. A Java backenddel szimbólikus futtatás is lehetséges, nem úgy mint a OCaml backenddel, viszont hátránya, hogy lassabb is. A keretrendszerhez sok más eszköz is tartozik, mint például a futásidőbeli verifikáló, a statikus és dinamikus tulajdonságokat ellenőrző szimbolikus bizonyító, melynek matematikai háttere a Matching Logic és a Reachibility Logic. A bizonyító nyelv független ellentétben az ismert Hoare logikával. Tehát nem kell minden nyelvre külön megalkotni a bizonyítót, csak a formális szemantikát kell megírnunk. A másik előnye pedig pont ez, mivel közvetlenül tudja használni a szemantikát, emiatt nem kell a szemantika és a bizonyító között semmiféle leképezést készíteni, amit természetesen növeli a rendszer sérülékenységét is.

\paragraph{}
Most hogy már megismerkedtünk felületesen a keretrendszer által nyújtott lehetőségeket, azt is hogy hogyan kell formális nyelvdefiníciót adni, megismerkedhetünk az Erlang nyelvvel is.