\section{$\mathbb{K}$ keretrendszer}

Diplomamunkám témája formális szemantikát definiálni az Erlang egy résznyelvéhez nem a megszokott módszerekkel, mint például denotációs vagy operációs szemantika, hanem a $\mathbb{K}$ keretrendszer segítségével. Ez a keretrendszer képes a szemantikai szabályok alapján egy értelmezőt készíteni, amivel programjaink futtathatóak, így ténylegesen megtekinthetjük az általunk definiált szemantikánk működését. Ezen felül még sok más érdekes funkcióval is rendelkezik. A keretrendszer fejlesztését Grigore Rosu kezdte 2003-ban \cite{KLecture}. Jelenleg az amerikai Illinois Urbana-Champaign Egyetem és a román Alexandru Ioan Cuza Egyetem közös projektje.

Az egyik ok, amiért ez a projekt létre jött, az eszközök hiánya formális szemantika definiálására. Ezenfelül ha definiálunk is valamilyen más módszerrel, nem lehet egyszerűen értelmezőt létrehozni, nem lehet helyesség bizonyítani programjainkat közvetlenül a szemantika alapján, a létrehozott modellt nem tudjuk ellenőrizni, hogy megfelel-e az adott specifikációnak és még sok minden felsorolható. Ha van egy formális nyelvdefiníciónk, akkor elvben az előbb felsoroltak megvalósíthatóak közvetlenül a nyelvdefiníciót használva, és nem kellene teljes mértékben támaszkodnunk a sok esetben ad-hoc módon implementált fordítókra. Ezt a célt tűzték ki a keretrendszer fejlesztői. A formális nyelv definiálására létrehozott módszer nem rendelkezik a már ismertek hátrányaival, mint például a modularitási vagy verifikálási nehézségek.

\subsection{Szemantika definiálása}

A keretrendszerrel teljes formális nyelvdefiníció készíthető, tehát nem csak a szemantika, hanem a szintaxis megadása is kötelező mégpedig Backus–Naur-formában (BNF). Ezekhez különböző attribútumokat társíthatunk. Közülük a legfontosabb a \textit{strict}, amely a kiértékelési stratégiát határozza meg. Az ilyen fajta megadási módszer kézenfekvő, hisz szemantikaszabályok esetén már csak a kiértékelt értékekkel kell foglalkozni.

\begin{greyBox}

\begin{syntaxBlock}{\nonTerminal{\sort{Exp}}}
  \syntax
    {\nonTerminal{\sort{Exp}} = \nonTerminal{\sort{Exp}}}
    {\kattribute{strict(2)}}
\end{syntaxBlock}

\end{greyBox}

A fenti példában az értékadás operátor szintaxisa látható BNF jelöléssel, és a hozzátartozó kiértékelési stratégiával. Az operátor fajtája \textit{Exp} és a két operandusé is ugyanaz, vagyis az értékadás mind két oldalán egy kifejezés áll, és az értékadás önmagában is egy kifejezés. A \textit{strict} attribútum után zárójelben egy kettes van, így a meghatározott kiértékelési stratégia: csak a második operandust értékeljük ki, így az értékadó operátorhoz tartozó szemantikai szabály csak ezután alkalmazható.

\begin{greyBox}

\begin{kblock}

\kconfig{
\kall{yellow}{T}{
  \kall{\K_COLOR}{k}{
    \variable[Pgm]{\$PGM}{user}
  }
  \mathrel{}
  \kall{\DEFS_COLOR}{defs}{
    \dotCt{Map}
  }
  \mathrel{}
  \kall{\ENV_COLOR}{env}{
  	\dotCt{Map}
  }
}}

\end{kblock}

\end{greyBox}

A keretrendszerben a definiált nyelvhez tartozik egy konfiguráció, ami egymásba ágyazott cellákból áll a sorrend figyelembe vétele nélkül. Ezek tartalmazzák az összes információt, amire a programnak, illetve nekünk elemzés szempontjából szükségünk van. A konfiguráció az állapotot tartalmazza, illetve egy speciális \textit{k} cellát, vagyis a számításokat, ami gyakorlatilag az absztrakt szintaxis fa kiegészítve egy lista struktúrával ami a $\sim>$ jelet használja mint szeparátor. Elemei speciális \textit{K} fajtájúak. A listára bontást a keretrendszer automatikusan elvégzi a \textit{syntax} kulcsszó alapján. Ennek eredménye hogy a szemantikaszabályok egy egységes \textit{K} fajtákat (\textit{term}) tartalmazó számítási folyamra tudnak illeszkedni.

\begin{greyBox}

\begin{kblock}

\krule{
    \kprefix{\K_COLOR}{k}{
      \reduce
        {\variable[]{X}{} \mathrel{=} \variable[]{V}{}}
        {\variable[]{V}{}}
    }
    \mathrel{}
    \kmiddle{\ENV_COLOR}{env}{
      \variable[]{X}{} \mapsto
      \reduce
        {\terminal{\_}}
        {\variable[]{V}{}}
    }
  }
  {}{}{}{}
%
<k> X = V => V ...</k>
\newline
<env>... X |-> (\_ => V) ...</env>
         
\end{kblock}         
         
\end{greyBox}

Az értékadó operátor $\mathbb{K}$ keretrendszerben kifejezett szemantikadefiníciója látható a fenti példán. Felül az olvashatóbb, prezentációra szánt változat, melyet generáltatni lehet a keretrendszerrel, illetve alul a ténylegesen írott forma látható. Két cellát a \textit{k}-t és a az \textit{env}-et tartalmazza. Ha a kiértékelési stratégia befejeződött, akkor alkalmazható. Ezek a szabályok a redukciós szabályok, annak ellenére, ha nem is csökkentik a konfigurációt. Az átmenet a \textit{k} cellában az értéket tartja meg, és ezalatt az \textit{env} cellában az \textit{X} $\mathbb{K}$ változóban lévő programbeli változóhoz hozzárendeli az értéket. Az átmenetben látható a \textit{\_} joker karakter, ami mindenre tud illeszkedni, tehát akármi is volt ehhez a változóhoz hozzárendelve, azzal nem foglalkozunk többet.

A szabály csak két cellát tartalmaz, azokat amik az értékadó operátor redukciós szabályában szerepet játszanak, a többivel nem foglalkozik. Ennek előnye, hogy a konfiguráció könnyedén bővíthető úgy, hogy nem kell a már meglévő szabályhalmazunkon módosítani. Az alsó résznél három pont látható a \textit{k} cella jobb, illetve az \textit{env} cella mindkét oldalán. Ezzel megadhatjuk hogy a cella elejére vagy végére akarunk illeszteni, esetleg ez számunkra lényegtelen. A szabályokhoz tartozhatnak különböző feltételek, melyek teljesülése esetén, és csak is akkor, illeszkedhetnek a szabályok a konfigurációra. A keretrendszer rendelkezik beépített típusokkal, ilyenek a teljesség igénye nélkül: halmaz, map, lista. Az \textit{env} cella, ahogy látható is volt a konfigurációban, egy mapben tárolja a változókhoz rendelt értéket.

Térjünk vissza a \textit{strict} attribútumhoz, hogy megértsük pontosan hogyan is működik. Példaként vegyük megint az értékadás operátort. A hozzátartozó attribútum az alábbi két szabályt generálja.

\begin{greyBox}

\begin{kblock}

\krule{
    \reduce
      {\variable[]{X}{} \mathrel{=} \variable[]{A}{};}
      {\variable[]{A}{} \kra \variable[]{X}{} \mathrel{=} [];}
  }
  {}{}{}{}
%
\krule{
    \reduce
      {\variable[]{A}{} \kra \variable[]{X}{} \mathrel{=} [];}
      {\variable[]{X}{} \mathrel{=} \variable[]{A}{};}
  }
  {}{}{}{}
         
\end{kblock}
         
\end{greyBox}

Itt megjegyezném, hogy a $\kra$ a $\sim>$ jelnek a prezentációra szánt formája. Az első szabály kiveszi az értékadó operátor kontextusából a második operandust, és berakja a folyam elejére egy új termként. A második pedig a kivett operandust visszarakja az értékadó operátor környezetébe. A kémiai absztrakt gép alapján ezeket rendre fűtő és hűtő szabályoknak nevezzük. \textit{strict} attribútum esetén a kiértékelési sorrend nem számit.

\begin{greyBox}

\begin{kblock}

\krule{
    \reduce
      {\variable[]{A1}{} \mathrel{+} \variable[]{A2}{}}
      {\variable[]{A1}{} \kra [] \mathrel{+} \variable[]{A2}{}}
  }
  {}{}{}{}
%
\krule{
    \reduce
      {\variable[]{A1}{} \kra [] \mathrel{+} \variable[]{A2}{}}
      {\variable[]{A1}{} \mathrel{+} \variable[]{A2}{}}
  }
  {}{}{}{}
%
\krule{
    \reduce
      {\variable[]{A1}{} \mathrel{+} \variable[]{A2}{}}
      {\variable[]{A2}{} \kra \variable[]{A1}{} \mathrel{+} []}
  }
  {}{}{}{}
%
\krule{
    \reduce
      {\variable[]{A2}{} \kra \variable[]{A1}{} \mathrel{+} []}
      {\variable[]{A1}{} \mathrel{+} \variable[]{A2}{}}
  }
  {}{}{}{}
         
\end{kblock}         
         
\end{greyBox}

Itt látható szabályok abban az esetben generálodnak, ha az összeadás operátor kiértékelési stratégiája olyan, hogy mind a két operandust ki kell értékelni az összeadásra vonatkozó szemantikus szabály alkalmazása előtt. Könnyen látható, hogy előidézhetnek nem determinisztikus futást. Ugyan ezek generálodnak a \textit{seqstrict} attribútum esetén is, de itt már számít a sorrend. Tehát az \textit{A2}-re egészen addig nem alkalmazza a fűtő/hűtő szabályokat, amíg az A1-re nem alkalmazta ezeket. Viszont a szabályok felcserélhetőek, aminek következménye, hogy a $\mathbb{K}$-ban definiált nyelv nem lesz futtatható, mivel előfordulhat hogy nem terminál. Ennek megelőzése érdekében lett bevezetve a \textit{KResult} fajta és a hozzá tartozó \textit{isKResult} szemantikus függvény, amely eldönti egy termről, hogy \textit{KResult} fajta vagy sem. Így a teljes generált szabályok értékadó operátor esetén az alábbiak:

\begin{greyBox}

\begin{kblock}

\krule{
    \reduce
      {\variable[]{X}{} \mathrel{=} \variable[]{A}{};}
      {\variable[]{A}{} \kra \variable[]{X}{} \mathrel{=} [];}
  }
  {\mathrel{\neg_{\scriptstyle\it Bool}} \terminal{isKResult}(\variable[]{A}{})}{}{}{}
%
\krule{
    \reduce
      {\variable[]{A}{} \kra \variable[]{X}{} \mathrel{=} [];}
      {\variable[]{X}{} \mathrel{=} \variable[]{A}{};}
  }
  {\terminal{isKResult}(\variable[]{A}{})}
  {}{}{}
         
\end{kblock}         
         
\end{greyBox}

Ilyenkor minden esetben először a fűtő szabályt alkalmazza, és egészen addig, amíg a kontextusból kiemelt term nem \textit{KResult} fajtájú, nem tudja alkalmazni rá a hűtő szabályt.

\subsection{Keretrendszer további eszközei}

Az előző alfejezetben láthattuk, hogyan definiálható egy formális nyelv a $\mathbb{K}$ keretrendszerrel. A kész definícióhoz a \textit{kompile} paranccsal készíthető értelmező, és a \textit{krun} paranccsal futtatható programokon. Alapértelmezett beállításokkal az előbb említett működés a mérvadó, viszont konfigurálhatjuk úgy, hogy az összes lehetséges kimenetelt megadja. Ez főleg konkurrens programoknál fordulhat elő. Tipikusan jó példa erre, mikor két szál egy globális változót akar módosítani. Az eredmény mindig függeni fog attól, hogy a gép hogyan ossza ki az órajelet a különböző szálaknak.

A keretrendszer három különböző backenddel van megvalósítva. Az elsőnek implementált verzióból lett a Maude nevű. A Java backenddel szimbolikus futtatás is lehetséges, nem úgy mint a OCaml backenddel, viszont hátránya, hogy lassabb is. A keretrendszerhez sok más eszköz is tartozik, mint például a futásidőbeli verifikáló, a statikus és dinamikus tulajdonságokat ellenőrző szimbolikus bizonyító, melynek matematikai háttere a Matching Logic és a Reachibility Logic. A bizonyító nyelvfüggetlen ellentétben az ismert Hoare logikával. Tehát nem kell minden nyelvre külön megalkotni egy modellt a bizonyításhoz, csak a formális szemantikát kell megírnunk. A másik előnye pedig pont ez, mivel közvetlenül tudja használni a szemantikát, emiatt nem kell a szemantika és a bizonyításkor használt modell között semmiféle leképezést készíteni, ami természetesen növelné a rendszer sérülékenységét is.

\paragraph{}
Most hogy már megismerkedtünk felületesen a keretrendszer által nyújtott lehetőségekkel, főként hogy hogyan kell formális nyelvdefiníciót adni, továbbléphetünk Erlang nyelv világába.

%Eddig OK