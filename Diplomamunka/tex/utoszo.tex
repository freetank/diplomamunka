\section{Összefoglalás}
A diplomamunkában olvashattunk a keretrendszerről, mely az operációs szemantikában megfogalmazott formális definíciókra támaszkodva képes interpretert készíteni, áttekintettük a meglévő szemantikadefiníciót, bevezetést kaphattunk az Erlang folyamatok világába, ezután a konkurens résznyelv szemantikájába ástuk be magunkat megvalósítva a $\mathbb{K}$ keretrendszer segítségével, és végül az ehhez tartozó tesztekről hallhattunk.

A témabejelentőben kitűzött célok nagy részét sikerült teljesítenem. Hiányzik az \textit{erlang:error} egy és két paraméterű változatának definíciója. Azonban a működésük hasonló a már definiált egy paraméterű \textit{exit} függvényhez. Az \textit{erlang:error} csupán kiegészíti a működését azzal, hogy nem csak a terminálás okát, hanem az aktuális folyamat vermét is elküldi a szignálban.

A monitorozásnak maradt még ki a szemantikadefiníciója. Különbség a folyamatok közötti kapcsolatokkal, hogy ez csak egyirányú, tehát tényleg egy folyamat monitoroz egy másikat és fordítva nem. Ennek megvalósítása könnyebb, mint a folyamatok közötti kapcsolatoké, az egyirányúság miatt. Mindig üzenetet küld a monitorozó folyamatnak ha terminál, nem foglalkozik a \textit{trap\_exit} flaggel és nem terminál a megfigyelő folyamat, ha a monitorozott leáll.

A cél a teljes formális nyelvdefiníció megírása, ami hosszadalmas folyamat. A jelen diplomamunka az elosztott esetekkel nem foglalkozott, amivel a \textit{CONCURRENT} modul fog  bővülni. Felmerül a kérdés, hogy a hardveres okokból eredő üzenetek késésének kezelése hogyan jelenik meg a szemantikadefinícióban. Ezenkívül külön függvények is tartoznak az elosztott programozás témájához.

Időközben egy másik ágon elkészült az Erlang modulra bontás definíciója is, amit majd össze kell fésülni az én munkámmal, figyelve milyen elemek kerültek be a konfigurációba, hogy a hasonló vagy ugyanazt kifejező állapotok eliminálva legyenek. Azonban a meglévő szemantikadefinícióval is elkezdődhet kisebb programok verifikálása. Igyekeztem a legnagyobb precizitással és körüljárással értelmezni a függvények pontos működését, majd formális szabályok formájába önteni, melynek eredménye ez a diplomamunka lett.