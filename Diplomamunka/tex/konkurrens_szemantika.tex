\section{Erlang nyelv konkurrens résznyelvének szemantikája K keretrendszerben}
A diplomamunka lényegi része az Erlang nyelv konkurrens résznyelvének szemantikájának definiálása volt. A K keretrendszernek erős kifejező ereje, megkönnyítette a szemantikadefiníció megadását, még ilyen bonyolult témában is, ahol különböző folyamatok kommunikálnak egymással. A konkurrens résznyelv szemantikájának leírását nem kellett a nulláról kezdeni, már meg volt egy mag, amire lehetett építeni. Ez a mag főleg az egyszerűbb kifejezéseket, függvénydefiniálások és - hívások szemantikáját tartalmazta. Az Erlang folyamatok ismertetése után megláthatjuk, hogyan épült fel erre az alapra a konkurrens résznyelv szemantikája.

\subsection{Meglévő szemantikadefiníciók}
Az első lépés a meglévő szemantikadefiníciók 3.6-os verzióról 4.0-ra való átírása, és ezek modularizálása.

\paragraph{tokens.k}
Ebben a fájlban van az Erlang specifikus literáloknak és a változóknak a szintaxisa, mely két modult tartalmaz.

\begin{module}{\moduleName{TOKENS-PARSING}}

\begin{syntaxBlock}{\nonTerminal{\sort{UnquotedAtom}}}
  \syntaxKAttrBox
    {r"[a-z][\_a-zA-Z0-9@]*"}
    {\kattribute{token}, \newline \kattribute{autoReject}, \newline \kattribute{notInRules}} %TODO nem fér ki egy sorba
\end{syntaxBlock}

\begin{syntaxBlock}{\nonTerminal{\sort{Atom}}}
  \syntax
    {{{\nonTerminal{\sort{UnquotedAtom}}}}}
    {}
  \syntaxCont
    {{{\nonTerminal{\sort{Bool}}}}}
    {}
\end{syntaxBlock}

\begin{syntaxBlock}{\nonTerminal{\sort{Joker}}}
  \syntaxNoEq
    {}
    {}
\end{syntaxBlock}

\begin{syntaxBlock}{\nonTerminal{\sort{Variable}}}
  \syntax
    {{{\nonTerminal{\sort{Joker}}}}}
    {}
\end{syntaxBlock}

\end{module}

A \textit{TOKENS-PARSING}, ahogy látható felül, az \textit{UnquotedAtom}, \textit{Atom}, \textit{Joker} és a \textit{Variable} fajtákat tartalmazza. A definíciók maguktól értetendőek. A \textit{Bool} és az UnquotedAtom alfajtája az Atomnak, viszont az aposztrófok közé írt atom nem, mivel egyelőre nincs definiálva. A Joker és a Variable definíciója ebben a modulban nem látható. Ennek az az oka, hogy a K keretrendszer rendelkezik egy \textit{\#KVariable} fajtával, amelynek a definíciójában megadott reguláris kifejezés által meghatározott lehetséges karakterláncok halmaza és a hasonló módon Variable által meghatározott halmaz metszete nem üres, emiatt ha itt definiálnánk, parszolási hibát kapnánk.
\newline
\begin{greyBox}

\begin{module}{\moduleName{TOKENS-SYNTAX}}

\including{TOKENS-PARSING}

\begin{syntaxBlock}{\nonTerminal{\sort{Variable}}}
  \syntaxKAttrBox
    {r"[A-Z][\_a-zA-Z0-9@]*"}
    {\kattribute{token}, \newline \kattribute{autoReject}}
\end{syntaxBlock}

\begin{syntaxBlock}{\nonTerminal{\sort{Joker}}}
  \syntax
    {{}\terminal{\_}}
    {\kattribute{token}}
\end{syntaxBlock}

\end{module}

\end{greyBox}

A TOKENS-SYNTAX modulban megadva a definíciókat ezt a hibát elkerülhetjük. Ilyenkor a parszolás után kapott fán lévő levelek \#KVariable címkéit lecseréli a saját Variable fajtával.

\paragraph{exp-shared.k}
Az \textit{EXP-SHARED} modul az \textit{EXP} modulból -- később kerül ismertetésre -- leválasztott egység.

\begin{greyBox}

\begin{module}{\moduleName{EXP-SHARED}}

\begin{syntaxBlock}{\nonTerminal{\sort{Exp}}}
  \syntaxNoEq
    {}
    {}
\end{syntaxBlock}

\begin{syntaxBlock}{\nonTerminal{\sort{Exps}}}
  \syntax
    {List\{{\nonTerminal{\sort{Exp}}}, ","\}}
    {\kattribute{strict}}
\end{syntaxBlock}

\begin{syntaxBlock}{\nonTerminal{\sort{Match0}}}
  \syntax
    {{\nonTerminal{\sort{Exp}}} \terminal{->} {\nonTerminal{\sort{Exp}}}}
    {\kattribute{match0}}
\end{syntaxBlock}

\begin{syntaxBlock}{\nonTerminal{\sort{Match}}}
  \syntax
    {List\{{\nonTerminal{\sort{Match0}}}, ";"\}}
    {\kattribute{match0}}
\end{syntaxBlock}

$\ldots$

\end{module}

\end{greyBox}

Az \textit{Exp} -- a kifejezések fajtája -- deklarálása itt található. Szinte az összes modul használja az Exp és az Exps fajtákat, viszont vannak olyanok, ahol szükségtelen az egész Exp-hez tartozó szintaxist importálni. Ha a modulokban újra lenne deklarálva, az névütközéshez vezetne. Az EXP-SHARED modul ezt hivatott elkerülni.

Ezenfelül még definiálva lett az \textit{Exps} fajta, ami vesszővel elválasztott Exp-ek sorozata a strict attribútummal, mely a kötelező kiértékelést jelöli, sorrendet figyelmen kívül hagyva. A \textit{Match0} a case és az if kifejezések által is használt részkifejezések, amely a mintát és az utána lévő kifejezést tartalmazza, a \textit{Match} pedig ezek pontosvesszővel elválasztott sorozata. Fontos megjegyezni, hogy ezek a részkifejezések kiértékelése sorban történik. Vagyis mindig csak a mintaillesztés során a nyíl előtti rész, majd egyezés esetén pedig annak az "ágnak" a kifejezése értékelődik ki.

\paragraph{operators.k}
Az \textit{OPERATORS-PARSING} modulban bővítjük a kifejezések szintaxisát aritmetikai, összehasonlító és logikai operátorokkal.

\begin{module}{\moduleName{OPERATORS-PARSING}}

\including{TOKENS-PARSING}

\begin{syntaxBlock}{\nonTerminal{\sort{Exp}}}
  \syntax
    {{}\terminal{not}{{\nonTerminal{\sort{Exp}}}}}
    {\kattribute{strict}, \kattribute{arith}}
  \syntaxCont
    {{{\nonTerminal{\sort{Exp}}}}\terminal{*}{{\nonTerminal{\sort{Exp}}}}}
    {\kattribute{strict}, \kattribute{arith}}
  \syntaxCont
    {{{\nonTerminal{\sort{Exp}}}}\terminal{div}{{\nonTerminal{\sort{Exp}}}}}
    {\kattribute{strict}, \kattribute{arith}}
  \syntaxCont
    {{{\nonTerminal{\sort{Exp}}}}\terminal{rem}{{\nonTerminal{\sort{Exp}}}}}
    {\kattribute{strict}, \kattribute{arith}}
  \syntaxCont
    {{{\nonTerminal{\sort{Exp}}}}\terminal{+}{{\nonTerminal{\sort{Exp}}}}}
    {\kattribute{strict}, \kattribute{arith}}
  \syntaxCont
    {{{\nonTerminal{\sort{Exp}}}}\terminal{-}{{\nonTerminal{\sort{Exp}}}}}
    {\kattribute{strict}, \kattribute{arith}}
  \syntaxCont
    {{{\nonTerminal{\sort{Exp}}}}\terminal{<}{{\nonTerminal{\sort{Exp}}}}}
    {\kattribute{strict}, \kattribute{arith}}
  \syntaxCont
    {{{\nonTerminal{\sort{Exp}}}}\terminal{=<}{{\nonTerminal{\sort{Exp}}}}}
    {\kattribute{strict}, \kattribute{arith}}
  \syntaxCont
    {{{\nonTerminal{\sort{Exp}}}}\terminal{>}{{\nonTerminal{\sort{Exp}}}}}
    {\kattribute{strict}, \kattribute{arith}}
  \syntaxCont
    {{{\nonTerminal{\sort{Exp}}}}\terminal{>=}{{\nonTerminal{\sort{Exp}}}}}
    {\kattribute{strict}, \kattribute{arith}}
  \syntaxCont
    {{{\nonTerminal{\sort{Exp}}}}\terminal{==}{{\nonTerminal{\sort{Exp}}}}}
    {\kattribute{strict}, \kattribute{arith}}
  \syntaxCont
    {{{\nonTerminal{\sort{Exp}}}}\terminal{/=}{{\nonTerminal{\sort{Exp}}}}}
    {\kattribute{strict}, \kattribute{arith}}
  \syntaxCont
    {{{\nonTerminal{\sort{Exp}}}}\terminal{andalso}{{\nonTerminal{\sort{Exp}}}}}
    {\kattribute{strict}(1), \kattribute{arith}}
  \syntaxCont
    {{{\nonTerminal{\sort{Exp}}}}\terminal{orelse}{{\nonTerminal{\sort{Exp}}}}}
    {\kattribute{strict}(1), \kattribute{arith}}
\end{syntaxBlock}

\end{module}

Az \textit{OPERATORS} modulban pedig az ezekhez tartozó szemantikadefiníciók találhatóak. Mivel a szintaxis esetén megadtuk minden egyes operátornál a strict attribútumot, ezért a szabály alkalmazása várat magára egészen addig, amíg az operandusokat ki nem értékelte.

\begin{greyBox}

\begin{kblock}

\krule{
    \reduce
      {{\variable[K]{I1}{}}\terminal{div}{\variable[K]{I2}{}}}
      {{\variable[K]{I1}{}}\mathrel{\div_{\scriptstyle\it Int}}{\variable[K]{I2}{}}}
  }
  {{\variable[K]{I2}{}}\mathrel{{=}{/}{=}_{\scriptstyle\it Int}}{\constant[\#Int]{0}}}{}{}{}

\end{kblock}

\end{greyBox}

Ahogy a példában látható, az aritmetikai operátorok szemantikájának definiálása esetén nagy segítséget jelent a K keretrendszer beépített operátorai.

\paragraph{tuple.k}
Ebben a fájlban a tuple típussal kapcsolatos szintaxis és szemantika található. A \textit{TUPLE-PARSING} modul bővíti az Exp fajtát a tuple szintaxisával.

\begin{module}{\moduleName{TUPLE-PARSING}}

\including{EXP-SHARED}

\begin{syntaxBlock}{\nonTerminal{\sort{Exp}}}
  \syntax
    {\terminal{$\lbrace$} \nonTerminal{Exps} \terminal{$\rbrace$}}
    {\kattribute{strict}, \kattribute{tuple}}
\end{syntaxBlock}

\end{module}

A \textit{TUPLE} modul az OPERATORS modul kibővítése tuple specifikus összehasonlító operátorokkal.

\begin{kblock}

\krule{
    \reduce
      {{\{{\left({{\variable[Value]{X}{user}}\mathpunct{\terminalNoSpace{,}}{\variable[Values]{Xs}{user}}}\right)}\}}\terminal{<}{\{{\left({{\variable[Value]{Y}{user}}\mathpunct{\terminalNoSpace{,}}{\variable[Values]{Ys}{user}}}\right)}\}}}
      {{\left({{\variable[Value]{X}{}}\terminal{<}{\variable[Value]{Y}{}}}\right)}\terminal{orelse}{\left({{{\variable[Value]{X}{}}\terminal{==}{\variable[Value]{Y}{}}}\terminal{andalso}{{\{{\variable[Values]{Xs}{}}\}}\terminal{<}{\{{\variable[Values]{Ys}{}}\}}}}\right)}}
  }
  {{{{{}\terminal{count}({\variable[Values]{Xs}{}})}\mathrel{{=}{=}_{\scriptstyle\it Int}}{{}\terminal{count}({\variable[Values]{Ys}{}})}}\wedge_{\scriptstyle\it Bool}{{{}\terminal{count}({\variable[Values]{Xs}{}})}\mathrel{>_{\scriptstyle\it Int}}{\constant[\#Int]{0}}}}\wedge_{\scriptstyle\it Bool}{{{}\terminal{count}({\variable[Values]{Ys}{}})}\mathrel{>_{\scriptstyle\it Int}}{\constant[\#Int]{0}}}}{}{\kattribute{structural}}{}

%TODO túl hosszú

\end{kblock}

Egy strukturális átalakítás látható, ami jó példa arra, hogyan lehet felhasználni, a már meglévő szemantikadefinícióinkat újabbak definiálására.

\paragraph{list.k}

Az \textit{ERL-LIST-PARSING} modulban az erlangban használatos lista két fajta szintaxisa található.

\begin{module}{\moduleName{ERL-LIST-PARSING}}

\including{EXP-SHARED}

\begin{syntaxBlock}{\nonTerminal{\sort{Exp}}}
  \syntax
    {\terminal{[} \nonTerminal{\sort{Exps}} \terminal{]}}
    {\kattribute{strict}, \kattribute{list}}
  \syntaxCont
    {\terminal{[} \nonTerminal{\sort{Exps}} \terminal{|} \nonTerminal{\sort{Exp}} \terminal{]}}
    {\kattribute{strict}, \kattribute{list}}
\end{syntaxBlock}

\ldots

\end{module}

Az \textit{ERL-LIST} modul csak átalakítási szabályokat tartalmaz melyet a \textit{macro} attribútum jelöl. Ezeket a keretrendszer a legelső számítási lépés előtt elvégzi.

\begin{greyBox}

\begin{module}{\moduleName{ERL-LIST}}

\including{ERL-LIST-PARSING}
\including{OPERATORS}
\including{CONFIG}

\krule{
    \reduce
      {[{\dotCt{Exps}}\terminal{|}{\AnyVar[K]{}}]}
      {\constant[\#String]{"badlist"}}
  }
  {}{}{\kattribute{macro}}{}
%
\krule{
    \reduce
      {[{{\variable[K]{X}{}}\mathpunct{\terminalNoSpace{,}}{\variable[K]{Y}{}}}]}
      {[{\variable[K]{X}{}}\terminal{|}{[{\variable[K]{Y}{}}]}]}
  }
  {}{}{\kattribute{macro}}{}
%
\krule{
    \reduce
      {[{{\variable[K]{X}{}}\mathpunct{\terminalNoSpace{,}}{\variable[K]{Xs}{}}}\terminal{|}{\variable[K]{Y}{}}]}
      {[{\variable[K]{X}{}}\terminal{|}{[{\variable[K]{Xs}{}}\terminal{|}{\variable[K]{Y}{}}]}]}
  }
  {{\variable[K]{Xs}{}}\mathrel{\neq_K}{\dotCt{Exps}}}{}{\kattribute{macro}}{}
    
\end{module}

\end{greyBox}

\paragraph{errors.k}

Az \textit{ERRORS} modul a futásidejű hibák fajtáit tartalmazza.

\begin{greyBox}

\begin{module}{\moduleName{ERRORS}}

\including{EXP-SHARED}

\begin{syntaxBlock}{\nonTerminal{\sort{Error}}}
  \syntax
    {\terminal{\textdollar error\_badmatch}}
    {}
  \syntaxCont
  	{\terminal{\textdollar error\_badarg}}
  	{}
  \syntaxCont
  	{\terminal{\textdollar error\_noproc}}
  	{}
\end{syntaxBlock}

\begin{syntaxBlock}{\nonTerminal{\sort{Exp}}}
  \syntax
    {\nonTerminal{\sort{Error}}}
    {}
\end{syntaxBlock}

\end{module}

\end{greyBox}

\paragraph{value.k}

Előző modulokban látható, hogy szintaxis esetén egyes fajtákhoz strict attribútum van rendelve. Ennek a következménye, hogy a szemantikai szabályok esetén, csakis akkor illeszkedik egy minta, ha a keretrendszer a szabály előfeltételében szereplő összes fajtát a K specifikus \textit{KResult} fajtára átalakította. A \textit{VALUE} modulban ezeket gyűjtöttük össze a \textit{Value} fajtába.

\begin{greyBox}

\begin{module}{\moduleName{VALUE}}

\including{EXP-SHARED}
\including{TOKENS-PARSING}
\including{ERRORS}

\begin{syntaxBlock}{\nonTerminal{\sort{BasicValue}}}
  \syntax
    {\nonTerminal{\sort{Atom}}}
    {\kattribute{value}}
  \syntaxCont
    {\nonTerminal{\sort{Int}}}
    {\kattribute{value}}
  \syntaxCont
    {\nonTerminal{\sort{Bool}}}
    {\kattribute{value}}
\end{syntaxBlock}

\begin{syntaxBlock}{\nonTerminal{\sort{ListValue}}}
   \syntax
    {\terminal{[} \nonTerminal{\sort{Values}} \terminal{]}}
    {}
  \syntaxCont
    {\terminal{[} \nonTerminal{\sort{Values}} \terminal{|} \nonTerminal{\sort{Value}} \terminal{]}}
    {}
\end{syntaxBlock}

\begin{syntaxBlock}{\nonTerminal{\sort{Value}}}
  \syntax
    {\nonTerminal{\sort{BasicValue}}}
    {}
  \syntaxCont
    {\nonTerminal{\sort{ListValue}}}
    {}
  \syntaxCont
    {\terminal{$\lbrace$} \nonTerminal{\sort{Values}} \terminal{$\rbrace$}}
    {}
  \syntaxCont
    {\nonTerminal{\sort{Error}}}
    {}
\end{syntaxBlock}

\begin{syntaxBlock}{\nonTerminal{\sort{Values}}}
  \syntax
    {List\lbrace \nonTerminal{\sort{Value}}, ","\rbrace}
    {}
\end{syntaxBlock}

\begin{syntaxBlock}{\nonTerminal{\sort{Exp}}}
  \syntax
    {\nonTerminal{\sort{Value}}}
    {}
\end{syntaxBlock}

\end{module}

\end{greyBox}

\paragraph{exp.k}

Az \textit{EXP-PARSING} modul tartalmazza azon kifejezések szintaxisát, melyekhez lett szemantika definiálva.

\begin{module}{\moduleName{EXP-PARSING}}

\including{EXP-SHARED}
\including{TOKENS-PARSING}

\begin{syntaxBlock}{\nonTerminal{\sort{Exp}}}
  \syntax
    {\nonTerminal{\sort{Atom}}}
    {}
  \syntaxCont
    {\nonTerminal{\sort{Int}}}
    {}
  \syntaxCont
    {\nonTerminal{\sort{Variable}}}
    {}
  \syntaxCont
    {\nonTerminal{\sort{Atom}} \terminal{(} \nonTerminal{\sort{Exps}} \terminal{)}}
    {\kattribute{strict}(2), \kattribute{funcall}}
  \syntaxCont
    {\terminal{case} \nonTerminal{\sort{Exp}} \terminal{of} \nonTerminal{\sort{Match}} \terminal{end}}
    {\kattribute{strict}(1), \kattribute{case}}
  \syntaxCont
    {\terminal{if} \nonTerminal{\sort{Match}} \terminal{end}}
    {\kattribute{if}}
  \syntaxCont
    {\terminal{begin} \nonTerminal{\sort{Exps}} \terminal{end}}
    {\kattribute{block}}
  \syntaxCont
    {\nonTerminal{\sort{Exp}} \terminal{=} \nonTerminal{\sort{Exp}}}
    {\kattribute{strict}(2), \kattribute{matchexpr}}
  \syntaxCont
    {\terminal{(} \nonTerminal{\sort{Exp}} \terminal{)}}
    {\kattribute{bracket}}
\end{syntaxBlock}

\end{module}

Az \textit{EXP} modul pedig az ezekhez tartozó szemantikadefiníciókat tartalmazza.

\begin{kblock}

\krule{
    \kprefix{green}{k}{
      \reduce
        {\terminal{case} \variable[Value]{E}{} \terminal{of} \variable[Match]{M}{} \terminal{end}}
        {\terminal{matches} ( \variable[K]{E}{}, \variable[K]{M}{} ) \terminal{$\sim\!\!>$} \variable[K]{Rho}{}}
    }
    \mathrel{}%TODO ???
    \kall{blue}{env}{\variable[K]{Rho}{}}
  }
  {}{}{}{}

\end{kblock}

A kiemelt példában érdemes megfigyelni, ahogy a case szabálya környezetet vált. A régi környezetet, amit a \textit{Rho} nevű változó jelöl, átemeli a kiértékelési folyamba. Mikor a case kifejezés visszatérési értékét megkapjuk, akkor ugyan ebből a folyamból visszaolvassa a régi környezetet.

\paragraph{erl-parsing.k}

Az ERL-PARSING modul egyesíti az összes -PARSING végű modult.

\begin{module}{\moduleName{ERL-PARSING}}

\including{TOKENS-PARSING}
\including{EXP-PARSING}
\including{OPERATORS-PARSING}
\including{ERL-LIST-PARSING}
\including{TUPLE-PARSING}
\including{CONCURRENT-PARSING}

\begin{syntaxBlock}{\nonTerminal{\sort{FunCl0}}}
  \syntax
    {\nonTerminal{\sort{Atom}} ( \nonTerminal{Exps} ) \terminal{$->$} \nonTerminal{Exp} ;}
    {\kattribute{funcl0}}
\end{syntaxBlock}

\begin{syntaxBlock}{\nonTerminal{\sort{FunCl1}}}
  \syntax
    {\nonTerminal{\sort{Atom}} ( \nonTerminal{Exps} ) \terminal{$->$} \nonTerminal{Exp} .}
    {\kattribute{funcl1}}
\end{syntaxBlock}

\begin{syntaxBlock}{\nonTerminal{\sort{FunCl}}}
  \syntax
    {\nonTerminal{\sort{FunCl0}}}
    {}
  \syntaxCont
    {\nonTerminal{\sort{FunCl1}}}
    {}
\end{syntaxBlock}

\begin{syntaxBlock}{\nonTerminal{\sort{FunDefs}}}
  \syntax
    {\nonTerminal{\sort{FunCl}}}
    {}
  \syntaxCont
    {\nonTerminal{\sort{FunDefs}} \mathrel{} \nonTerminal{\sort{FunDefs}}}
    {\kattribute{right}}
\end{syntaxBlock}

\ldots

\begin{syntaxBlock}{\nonTerminal{\sort{Pgm}}}
  \syntax
    {\nonTerminal{\sort{FunDefs}} \terminal{$---$} \nonTerminal{\sort{Exp}} .}
    {}
\end{syntaxBlock}

\end{module}

Ezenfelül meghatározza a különböző kifejezések közötti prioritásokat is. Jelenleg az Erlang modul rendszer szintaxisa még nincs kidolgozva. Egy fájl felépítése két nagy részből áll: a függvénydefiníciós rész és a program rész, amit $---$ jel választ el egymástól.

\paragraph{erl.k}

Az \textit{ERL} modul három dologért felelős.

\begin{greyBox}

\begin{kblock}

\begin{syntaxBlock}{\nonTerminal{\sort{KResult}}}
  \syntax
    {\nonTerminal{\sort{Value}}}
    {}
  \syntaxCont
  	{\nonTerminal{\sort{Values}}}
  	{}
  \syntaxCont
  	{\nonTerminal{\sort{Error}}}
  	{}
\end{syntaxBlock}

\end{kblock}

\end{greyBox}

Először is a Value alfajtája lesz a \textit{KResult} K specifikus fajtának.

\begin{greyBox}

\begin{kblock}

\krule{
    \reduce
      {{\variable[FunDefs]{F}{user}}\terminal{$---$}{\variable[Exp]{E}{user}}\terminal{.}}
      {\variable[FunDefs]{F}{}\kra\variable[Exp]{E}{}}
  }
  {}{}{\kattribute{structural}}{}
%
\krule{
    \reduce
      {{\variable[FunCl]{F1}{user}}\mathrel{}{\variable[FunDefs]{F2}{user}}}
      {\variable[FunCl]{F1}{}\kra\variable[FunDefs]{F2}{}}
  }
  {}{}{\kattribute{structural}}{}
%
\kruleReqInBox{
    \kprefix{\K_COLOR}{k}{
      \reduce
        {{\variable[Atom]{Name}{user}}({\variable[K]{Args}{}}){}\terminal{->}{\variable[K]{Body}{}}\terminal{.}}
        {\dotCt{K}}
    }
    \newline
    \mathrel{}
    \kall{\DEFS_COLOR}{defs}{
      {\variable[Map]{{\rho}}{user}}\mathrel{}{
      \reduce
        {\dotCt{Map}}
        {{\variable[Atom]{Name}{}}\mapsto{{}\terminal{ListItem}({{\{{\variable[K]{Args}{}}\}}\terminal{->}{\variable[K]{Body}{}}})}}
      }
    }
  }
  {\neg_{\scriptstyle\it Bool}{{\variable[Atom]{Name}{}}\terminal{in}{{}\terminal{keys}({\variable[Map]{{\rho}}{}})}}}{}{\kattribute{structural}}{}
%
\kruleReqInBox{
    \kprefix{\K_COLOR}{k}{
      \reduce
        {{\variable[Atom]{Name}{user}}({\variable[K]{Args}{}}){}\terminal{->}{\variable[K]{Body}{}}\terminal{;}}
        {\dotCt{K}}
    }
    \newline
    \mathrel{}
    \kall{\DEFS_COLOR}{defs}{
      {\variable[Map]{{\rho}}{user}}\mathrel{}{
      \reduce
        {\dotCt{Map}}
        {{\variable[Atom]{Name}{}}\mapsto{{}\terminal{ListItem}({{\{{\variable[K]{Args}{}}\}}\terminal{->}{\variable[K]{Body}{}}})}}
      }
    }
  }
  {\neg_{\scriptstyle\it Bool}{{\variable[Atom]{Name}{}}\terminal{in}{{}\terminal{keys}({\variable[Map]{{\rho}}{}})}}}{}{\kattribute{structural}}{}
%
\kruleReqInBox{
    \kprefix{\K_COLOR}{k}{
      \reduce
        {{\variable[Atom]{Name}{user}}({\variable[K]{Args}{}}){}\terminal{->}{\variable[K]{Body}{}}\terminal{.}}
        {\dotCt{K}}
    }
    \newline
    \mathrel{}
    \kmiddle{\DEFS_COLOR}{defs}{
      {\variable[Atom]{Name}{}}\mapsto{
      \reduce
        {\variable[K]{Bodies}{}}
        {{\variable[K]{Bodies}{}}\mathrel{}{{}\terminal{ListItem}({{\{{\variable[K]{Args}{}}\}}\terminal{->}{\variable[K]{Body}{}}})}}
      }
    }
  }
  {}{}{\kattribute{structural}}{}
%
\kruleReqInBox{
    \kprefix{\K_COLOR}{k}{
      \reduce
        {{\variable[Atom]{Name}{user}}({\variable[K]{Args}{}}){}\terminal{->}{\variable[K]{Body}{}}\terminal{;}}
        {\dotCt{K}}
    }
    \newline
    \mathrel{}
    \kmiddle{\DEFS_COLOR}{defs}{
      {\variable[Atom]{Name}{}}\mapsto{
      \reduce
        {\variable[K]{Bodies}{}}
        {{\variable[K]{Bodies}{}}\mathrel{}{{}\terminal{ListItem}({{\{{\variable[K]{Args}{}}\}}\terminal{->}{\variable[K]{Body}{}}})}}
      }
    }
  }
  {}{}{\kattribute{structural}}{}

\end{kblock}

\end{greyBox}

Másodszor a program fájl feldolgozását definiálja. A függvénydefiníciós részt a későbbi függvény hívások miatt a \textit{defs} állapotba elmenti.

\begin{greyBox}

\begin{kblock}

\krule{
    \kprefix{green}{k}{
      \reduce
        {\variable[Atom]{Name}{}}
        {\terminal{matches} ( \{ \variable[]{Args} \}, \terminal{getMatch} ( \variable[List]{L}{} ) ) \kra \variable[]{\rho}{} }
    }
    \mathrel{}
    \kmiddle{blue}{defs}{
      \variable[Atom]{Name}{} \mapsto \variable[]{L}{}
    }
    \mathrel{}
    \kall{yellow}{defs}{
      \reduce
        {\variable[]{\rho}{}}
        {\dotCt{Map}}
    }
  }
  {}{}{}{}
%
         
\end{kblock}         
         
\end{greyBox}

Harmadszorra pedig az általunk definiált függvények hívásának szemantikáját definiálja. Itt a megszokott módon környezetet vált, és a defs állapotban megkeresi a hozzá tartozó függvénydefiníciót.

\paragraph{config.k}

Ebben a fájlban a kezdő konfiguráció található.

\begin{greyBox}

\begin{kblock}

\kconfig{
\kall{yellow}{T}{
  \kall{\K_COLOR}{k}{
    \variable[Pgm]{\$PGM}{user}
  }
  \mathrel{}
  \kall{\DEFS_COLOR}{defs}{
    \dotCt{Map}
  }
  \mathrel{}
  \kall{\ENV_COLOR}{env}{
  	\dotCt{Map}
  }
}}

\end{kblock}

\end{greyBox}

A fenti kezdő konfiguráció ezen diplomamunka előtti állapotot mutatja. A \textit{k} cella tartalmazza az adott konfigurációban a hátralévő programot. Az állapot jelenleg két parciális függvényből áll. 

\begin{align*}
defs: Atom \rightarrow Lista \quad \textrm{ahol} \quad Lista \subseteq \{f | f : Exps \rightarrow Exp\}
\end{align*}

Ahogy feljebb olvasható a \textit{defs} tartalmazza a függvénydefiníciókat: a függvény nevéhez hozzárendeli a lehetséges függvénytörzseket.

\begin{align*}
env: Variable \rightarrow Value
\end{align*}

Az \textit{env} függvény pedig a változó nevéhez hozzárendeli a változóhoz kötött értéket.

% Eddig ellenorizve

%TODO
%	Meglévő szemantikadefiníciók
%	ha kell külön subsection az erlang folyamatokról
%	szemantika definíció kidejtése, plusz mindegyiknél mit jelent a kifejezés
%	milyen tesztek voltak hogy működik mennyire segített, honnan veszem, hogy azok helyesek?
\subsection{Erlang folyamatok}