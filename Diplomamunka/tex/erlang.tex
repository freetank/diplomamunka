\section{Erlang nyelv}
Az \textit{Erlang} egy általános célra felhasználható funkcionális programozási nyelv. A Java nyelvhez hasonlóan egy szemétgyűjtővel rendelkező virtuális gépen fut a bájt-kódra lefordított program. Ez a fejezet egy rövid betekintést ad az Erlang nyelvhez a reference manual alapján \cite{RefMan}. A lent található példákhoz nem szükséges fájlba mentett, modularizált kódot használnunk, közvetlenül az Erlang Shell segítségével interaktívan is ki tudjuk értékelni kifejezéseinket.

A következő alfejezetek az Erlang nyelv alapjait mutatja be, éppen annyit, amennyi elégséges lesz a szemantikadefiníciók megértéséhez. Mielőtt elkezdnénk az adattípusok áttekintését, ismerkedjünk meg a \textit{term} fogalmával. A term a kifejezések legegyszerűbb formája, tehát az adattípusok. Minden kifejezés végét ellenben a megszokott pontosvesszős megoldással, pont jelöli.

\subsection{Adattípusok}
Első körben érdemes az egyszerűbb adattípusokat áttekinteni, mielőtt megismernénk más kifejezéseket.

Két típusú szám literál létezik: \textit{integer} és \textit{float}. Alapértelmezetten decimális számrendszerben értendők, de a \textit{alap\#érték} formában 2-36-os számrendszerig ki tudunk fejezni számokat.

\lstinputlisting[language=Erlang]{\EXAMPLESFOLDER/numbers.erl}

A leggyakrabban használt literál az \textit{atom}, gyakorlatilag egy konstans. Az aposztróf nélküli atomoknak kis betűvel kell kezdődniük és alfanumerikus karaktereket, aláhúzást és @ jelet tartalmazhatnak. Az aposztróffal rendelkező atomok Latin-1 karakterkódolású karakterből állhatnak.

\lstinputlisting[language=Erlang]{\EXAMPLESFOLDER/atoms.erl}

A \textit{tuple} egy fix n-es, vagyis előre meghatározott számú termet tartalmazhat. A termeket kapcsos zárójelek közé, vesszővel elválasztva adjuk meg. Természetesen egymásba is ágyazhatóak.

\lstinputlisting[language=Erlang]{\EXAMPLESFOLDER/tuples.erl}

A \textit{lista} is a tuplehöz hasonlóan egy összetett típus ellenben változó hosszúságú. A listákat szögletes zárójelekkel adjuk meg, és szintén egymásba ágyazhatóak. Kétféleképpen lehet megadni Erlang nyelvben: a termek vesszővel elválasztott formában, vagy a fej elem, maradék lista formában.

\lstinputlisting[language=Erlang]{\EXAMPLESFOLDER/lists.erl}

Egy érdekes adattípus, ami manapság már sok imperatív nyelvben is megtalálható, a \textit{függvény objektum}. A \textit{fun} szóval tudunk létrehozni ilyen objektumot.

\lstinputlisting[language=Erlang]{\EXAMPLESFOLDER/fun.erl}

A példában egy függvény objektumot hoztunk létre, amely két paramétert vár \textit{X} és \textit{Y}, és ezeknek az összegével tér vissza. Láthatjuk is a példában, hogy a visszatérési értéke az objektumra mutató referencia. A nagy betűvel kezdődő szavak Erlangban a \textit{változók}, később még lesz róluk szó. A \textit{Sum} változóban tárolt függvényt zárójeles formában meg is tudjuk hívni.

A \textit{pid} egy folyamatot azonosít három integer számmal.

\lstinputlisting[language=Erlang]{\EXAMPLESFOLDER/pid.erl}

A \textit{spawn} függvény segítségével egy új folyamatot hozunk létre, és visszatérési értékként egy pidet ad vissza.

Még mielőtt folytatnák érdemes szót ejteni a \textit{sztring} és a \textit{logikai} literálokról. Erlangban mint külön adattípusok nem léteznek. A két logikai literál valójában két atom: a \textit{true} és a \textit{false}. A sztringek pedig listák. Idéző jelek között tudjuk megadni, de valójában az ASCII vagy unicode kódtáblában megfeleltetett számok listájaként van ábrázolva.

\subsection{Mintaillesztés}
Az előző részben már volt említés a változókról. A \textit{változó} egy kifejezés. Ha egy literál kötve van a változóhoz, akkor a visszatérési értéke az a literál. Minden változóhoz csak egyszer lehetséges literált kötni, ami történhet mintaillesztés során, vagy egyszerű értékadással (ami szintén egy mintaillesztés). A változók nagy betűvel vagy aláhúzással kezdődnek, és alfanumerikus karaktereket, aláhúzást vagy @ jelet tartalmazhatnak. Az aláhúzással kezdődő változóknak speciális jelentésük van. A fordító figyelmen kívül hagyja olyan értelemben, hogy nem generál warningokat a nem használt változók miatt.

\lstinputlisting[language=Erlang]{\EXAMPLESFOLDER/_variables.erl}

A példákban mintaillesztés látható. A \textit{minta} ugyan úgy épül fel mint egy term, ellenben tartalmazhat nem kötött változókat. \textit{Mintaillesztés} során pedig ezekhez a változókhoz kötünk értékeket. Ha a mintaillesztés sikertelen, mint a második példában, akkor \textit{badmatch} futásidejű hiba lép fel. A változókhoz csak egyszer köthető érték, és ez alól nem kivétel az aláhúzással kezdődőek sem, attól függetlenül, hogy a fordító figyelmen kívül hagyja.

Az anonymus változó csak egyetlen egy aláhúzásból áll. Akkor hasznos, ha kötelező változót megadnunk, de a benne lévő érték nem fontos.

\lstinputlisting[language=Erlang]{\EXAMPLESFOLDER/anonymusvariable.erl}

Az első példa jól mutatja, hogy gyakorlatilag figyelmen kívül hagyja az értékeket. A második példában egy lista fejelemére vagyunk kiváncsiak, amit mintaillesztéssel egyszerűen megkaphatunk.

Mintaillesztés nem csak az egyenlőség jellel lehetséges, hanem más kifejezésekben is, mint például a \textit{receive} és a \textit{case}, amire később láthatunk példát.

\subsection{További kifejezések}

Az előzőekben egyszerű kifejezésekkel ismerkedtünk meg. Ez az alfejezet pedig olyan összetettebbeket mutat be, melyeknek már meg volt a $\mathbb{K}$ keretrendszer által kifejezett szemantikadefiníciója, mielőtt neki kezdtem volna a konkurrens résznyelvhez definiálni.

Az \textit{if} kifejezés merőben eltérő az imperatív nyelvekben megszokotthoz.

\lstinputlisting[language=Erlang]{\EXAMPLESFOLDER/if.erl}

Az \textit{if} kifejezés ágain sorban megy végig, és ahol az első őrfeltétel (guard) igazra értékelődik ki, vagyis true atom lesz a kifejezés értéke, annak az ágnak a kifejezését értékeli ki. A visszatérési értéke pedig az ágban lévő kifejezés értéke lesz.

\lstinputlisting[language=Erlang]{\EXAMPLESFOLDER/if2.erl}

A példában előfordulhatna \textit{badmatch} futásidejű hiba, ha az \textit{Animal} változó értékét megváltoztatnánk \textit{pig} atomra, hisz ekkor egyik ág őrfeltétele sem adna igazat.

\lstinputlisting[language=Erlang]{\EXAMPLESFOLDER/case.erl}

A \textit{case} esetén az \textit{Expr} kifejezés kiértékelődik, ezután sorban végig megy az ágakon, és mintaillesztést hajt végre. Ahol a mintaillesztés sikeres, és az őrfeltétel is igazra értékelődik ki, annak az ágnak a kifejezését kiértékeli. A \textit{case} kifejezés visszatérési értéke az ágban lévő kifejezés értéke. Szintén, ha az összes ágra sikertelen a mintaillesztés, akkor \textit{badmatch} futásidejű hibát kapunk.

\lstinputlisting[language=Erlang]{\EXAMPLESFOLDER/block.erl}

A \textit{block} kifejezések sorozata, melyek sorrendben értékelődnek ki, gyakorlatilag szekvenciát adunk meg. A visszatérési értéke az utolsó kifejezés értéke.

\subsection{Függvények}

A függvények deklarációja függvényklózók sorozata. Egy fejlécből áll, amely a függvény nevét, paramétereit tartalmazza és opcionálisan egy őrfeltételt, illetve egy függvénytörzsből.

\lstinputlisting[language=Erlang]{\EXAMPLESFOLDER/fuggveny_dekl.erl}

A függvény név egy atom a paraméterei pedig minták. Egy függvényt egyértelműen a modulnév a függvénynév és a hozzá tartozó aritása határozza meg.

Függvényhíváskor megpróbálja megkeresni a hivatkozott függvény deklarációját. Ha nem találja meg, akkor \textit{undef} futásidejű hibát ad. Ha megtalálja, az első függvényklóztól kezdődően a függvény fejléceire mintaillesztést végez. Ha az illesztés sikeres és az őrfeltétel is teljesül, ha létezik, akkor a függvénytörzset kiértékeli. Ha a mintaillesztés sikertelen, vagy sikeres illesztések esetén egy őrfeltétel sem teljesül, akkor \textit{function\_clause} futásidejű hibát kapunk.

\paragraph{}
Ebben a fejezetben egy rövid összefoglalót láthattunk az Erlang nyelvről. Ez az áttekintés megkönnyíti a későbbi szemantikadefiníciók megértését. Természetesen ez csak egy igen apró része a nyelvnek. Az hogy egy valós Erlang program hogy néz ki, nem tettem említést, mivel a szemantika definiálásánál egyelőre kifejezések állnak a központban, a modulokra bontás nem. Még az Erlang konkurrens részének ismertetése hátra van, de azt később a hozzájuk tartozó szemantikadefiníciók ismertetésénél részletezem.