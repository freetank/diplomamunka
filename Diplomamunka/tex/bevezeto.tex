\section{Bevezetés}
Az idők folyamán egyre több nyelv jelent meg a szoftverfejlesztés területén, mely vagy egy adott problémakörre specializálódtak, vagy általános felhasználásra készültek. Amióta megjelentek a magasabb szintű programozási nyelvek, azóta szigorú nyelvtani szabályok alapján adható meg a program, mely hibák fordítási időben kiderülnek. Mostanra kidolgozott elméleti háttérrel rendelkeznek és egy újonnan létrejövő nyelvnél nem is kérdéses a formális szintaxis kidolgozása. Oka érthető, hiszen ezek alapján a szabályok alapján parszolják a kódot és építenek szintaxis fát, ami elengedhetetlen része egy fordítónak. Nagyobb probléma viszont a szemantikai hiba, melyet futásidőben veszünk észre. Gyakorlatban az idetartozó definíciókat informálisan szöveges leírással adják meg, annak ellenére, hogy a formális szemantika definiálására is kialakult már biztos elméleti háttér. Ennek kiküszöbölésére készülnek nagy mennyiségben tesztek, melyek próbálják kisebb-nagyobb sikerrel lefedni az összes lehetséges működést, és így a szemantikai hibákra fényt deríteni.

Másik megközelítés viszont a formális szemantikadefiníció készítése lehetne. Az informális definíciók nem pontosak, félreértésekhez vezethetnek. Mivel közvetlenül nem szokás formális szemantikai szabályokat használni fordítók írásakor, nehézkes is lenne, ezért nem készítenek. Ezenfelül egy-egy vezérlési szerkezetnek nagyon bonyolult, nehezen megfogalmazható szemantikadefiníciója van, ami még egy okot ad arra a nyelv készítőknek, hogy ne foglalkozzanak formális szemantika definiálásával. Sokszor még a nyelv fejlesztői között sincs egyetértés, hogy egy bonyolult de rövid programrészletnek mi az eredménye. Szükség van formális szemantikadefinícióra, és nem csak akadémiai körökben. Nélküle nem lehetne programhelyesség bizonyítást végezni, a kérdéses esetekben az informális definíciók nem tudnak kielégítő választ adni. Sőt mi több előfordulhatnak ellentmondások, vagy olyan szemantikai szabályok, ami a többi szabály miatt nem aktiválható.

A meglévő formális szemantikadefiníciókat viszont nehézkes alkalmazni közvetlenül az előbb felsoroltakra egységesen. Ezt próbálja meg kiküszöbölni a $\mathbb{K}$ keretrendszer, amivel operációs szemantikával megadhatjuk egy nyelvnek a formális szemantikáját, és a hozzátartozó többi eszköz segítségével közvetlenül tudunk a szemantika felhasználásával verifikálni.

A felsorolt problémák és a keretrendszer lehetőségei az, ami ösztönzött diplomamunkám elkészítésére. A következő fejezetekben az Erlang nyelv párhuzamos programozását támogató résznyelvének formális szemantikadefiníciója rajzolódik ki előttünk. Sikerült nagy részét lefednem a definícióval, ám kisebb hiányosságok maradtak, mint például a monitorozás, ám a folyamatok közti kapcsolat egymás felügyelésére megvalósult, amely jó kiindulást adhat ennek befejezésére.
%TODO
%	Kicsit lemásolni az előadásban mondottakat