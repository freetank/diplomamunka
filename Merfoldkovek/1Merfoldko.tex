% Preamble
% ---
\documentclass[]{article}
% \documentclass[11pt,a4paper]{article}

% Packages
% ---
\usepackage[utf8]{inputenc}
\usepackage[magyar]{babel}
\usepackage[T1]{fontenc}

\usepackage{xcolor}
\usepackage{listings}

\begin{document}

\section*{1. Mérföldkő}
Az első mérföldkőig az Erlang nyelv párhuzamos programozást támogató résznyelvének K-ban definiált futtatható formális szemantikájának elkészítése volt a cél. Emiatt a már meglévő szemantika definíció bővült egy CONCURRENT-PARSING és egy CONCURRENT modullal, és ennek megfelelően a konfiguráció is bővült. Továbbá 38 darab teszt segítségével a munka során folyamatosan tudtam ellenőrizni szemantika definícióm helyességét.

\subsection*{A konfiguráció bővülése}
A jelenlegi konfiguráció alább látható:

\lstset{language=XML}
\begin{lstlisting}
<T color="yellow">
  <processes color="orange">
    <process multiplicity="*" color="magenta">
      <k color="green"> $PGM:Pgm </k>
      <env color="LightSkyBlue"> .Map </env>
      <pid> <0.0.0> </pid>
      <mailbox> .List </mailbox>
      <processFlag> trap_exit |-> false </processFlag>
      <terminatedProcess> false </terminatedProcess>
    </process>
  </processes>
  <processIds> SetItem(<0.0.0>) </processIds>
  <processesCheckMails> .Set </processesCheckMails>
  <idleProcesses> .Map </idleProcesses>
  <processesDuringExit> .Set </processesDuringExit>
  <registeredProcesses> .Map </registeredProcesses>
  <linkedProcesses> .Map </linkedProcesses>
  <time> 0 </time>
  <defs color="blue"> .Map </defs>
</T>
\end{lstlisting}

A processes tartalmazza, az összes futó folyamat konfigurációját. Itt négy típusértékhalmazzal bővült az állapot tér, ami a párhuzamos résznyelvhez kötődik.

\begin{itemize}
\item pid: A folyamatazonosítóját tartalmazza. A fő folyamatazonosítója mindig <0.0.0> lesz. Szükséges itt beállítani, mivel nem lehet úgy véletlenszerűen generáltatni, hogy azt a processIds állapot is tudja tartalmazni.
\item mailbox: Egy lista, amely folyamatnak küldött üzeneteket tartalmazza. Mivel listában van tárolva a sorrendiség megmarad.
\item processFlag: A process\_flag függvény által beállítható flageket tartalmazza. Csak a trap\_exit flaggel foglalkoztam a szemantika definiálás során.
\item terminatedProcess: Egy logikai értéket tartalmaz, ami jelzi, hogy a folyamat terminált-e vagy sem.
\end{itemize}

A többi állapot közös az összes folyamatra nézve.

\begin{itemize}
\item processIds: Az összes futó folyamatazonosítója. Ennek segítségével a jövőben eliminálható a terminatedProcess típusértékhalmaz.
\item processesCheckMails: Azon folyamatok azonosítója, melyek éppen receive utasításnál vannak, és éppen vizsgálják a mailboxukban lévő üzeneteket.
\item idleProcesses: Azok a folyamatok azonosítója, melyek receive utasításnál vannak, és már ellenőrizték az üzeneteiket. Emiatt újabb üzenetekre várnak, illetve ha rendelkeznek after résszel, akkor az ott meghatározott idő letelésére is figyelnek.
\item processesDuringExit: Azon folyamatok azonosítója, melyek exit szignált kaptak, vagy elérték a program végét, és az összelinkelt folyamataiknak jelzik, hogy terminálnak.
\item registeredProcesses: Névvel regisztrált folyamatok.
\item linkedProcesses: Az egymással link függvénnyel összekapcsolt folyamatok.
\item time: A program elindulásától számított eltelt időt reprezentálja. Mivel a K keretrendszertől nem lehet az adott időt lekérdezni, emiatt minden egyes utasítás egységesen 10-el növeli az értéket, ami milliszekundumban értendő.
\end{itemize}


\subsection*{Szemantika definíció}
Nem a párhuzamos programozást támogató résznyelvhez tartozik, de a mintaillesztés definícióján is finomítottam. Eddig csak változókhoz lehetett értéket rendelni.

\subsubsection*{Folyamat létrehozása}
Új folyamatot a spawn függvénnyel tudunk létrehozni. Az állapottér processes részébe új process állapot kerül, generáltatva magának egy egyedi folyamatazonosítót.

\subsubsection*{Regisztrált folyamatok}
A register utasítással tudunk nevet adni egy folyamatnak. Ekkor az állapottér registeredProcesses állapotában lévő map-be teszi a megadott név és a folyamatazonosítójából készített párt. Ha már egy névhez van regisztrálva név, vagy az adott névhez már regisztráltak nevet, akkor bad\_arg hibával tér vissza a függvény. Ezenfelül ugyanezt a működést érjük el, ha a megadott név undefined, vagy egy nem meglévő folyamatot akarunk regisztrálni.

A registered és a whereis utasítások ebből a registeredProcesses állapotból tudják lekérni a számukra szükséges adatokat. Az unregiszter pedig törli ebből állapotból a névhez tartozó értéket.

\subsection*{Üzenetküldés}
A time típusértékhalmaz a receive utasítás after klóza miatt került be. A receive hívás után a folyamat két állapotba kerülhet. Legelőször mintaillesztés fut le a mailbox állapotban lévő üzenetekre. Ekkor a folyamatazonosítója bekerül a processesCheckMails állapotba. Ha nem volt megfelelő illesztés, akkor várakozni kezd, kikerül a processesCheckMails-ből az azonosító és az idleProcessesbe kerül. Itt a legnehezebb rész az volt, amikor minden folyamat várakozik. Ilyenkor egy szabály ellenőrzi, hogy ez az állapot van-e, és folyamatosan növeli az időt. Ha volt after klóz, akkor egy idő után az egyik folyamat tud folytatódni.

A ! operátor egyszerűen csak a jobb oldalán lévő kiértékelt értéket beteszi annak a folyamatnak a mailbox-ába, amelynek az azonosítója az operátor bal oldali operandusa. Külön szabályokat kellett hozni az éppen várakozó folyamatoknak való küldéshez, illetve a regisztrált nevek esetén.

\subsubsection*{Linkek}
A link függvény összeköti a két folyamatot (a hívót és a paraméterben átadott folyamatazonosítóhoz tartozó folyamatot), amit linkedProcesses állapotban tárol map formájában. Mivel mindegyik folyamathoz több folyamat linkelhető, ezért minden kulcshoz a map-ben egy halmaz tartozik. A linkelés két lépésben történik. Először a link függvény bejegyzi a hívó folyamatazonosítóját, és hozzárendeli az argumentumot, ezután az állapot átmenet a \$link függvényhez vezet, ami pedig az argumentumot jegyzi be a linkedProcess állapotba.

Az unlink ebből az állapotból törli az egyes kulcsokhoz tartozó halmazból a folyamatok azonosítóit, de a kulcsot nem törli, még ha a halmaz üres is lesz.

A spawn\_link első ránézésre egy strukturális átrendezésnek tűnik, ami a link(spawn(...)) lenne. Sajnos a link visszatérési értéke true, a spawn\_link függvényé pedig a létrehozott folyamatazonosítója. A két lépéses link miatt, egy zárt \$spawn\_link nevű környezetbe raktam, ahol a spawn kiértékelése után, nem dobja el a folyamatazonosítót, hanem a link kiértékelése után visszatérési értékként megadja.

\subsubsection*{Folyamat terminálása}
Ha egy folyamat hiba nélkül lefutott, akkor a linkedProcesses-ben megadott folyamatoknak normal üzenetet küld, hiba esetén viszont az adott hibát küldi el. Ez az üzenet küldés megegyezik az exit függvénnyel. Ez egy \$exit függvényt hív meg, amely egyesével végig megy a folyamatazonosítókon, és attól függően, hogy a trap\_exit flag be van állítva, az alapján cselekszik.

\paragraph*{}
A definiált formális szemantika egy node-s környezetnek megfelelő működést írja le. Több node-s esetben számolni kéne azzal, hogy az üzenet nem egyből érkezik meg, tehát generálni kéne minden üzenetküldéshez egy időt, ami a hálózat sebességét írja le. Ezenkívül plusz függvények szemantikájának definiálását is megkívánja a több node-os eset, illetve a már meglévők kibővítését is.

Ahogy a match esetén, a receive esetén sem működik kifejezésekre a mintaillesztés, mivel nem értékeli ki, és hiányoznak a guardok is.


\subsection*{Teszt}
A tesztek egy kis részhalmaza esetén a kimenet függhet az új szemantika definícióktól. Ezek olyan tesztek, amik a regisztrált folyamatazonosítót ellenőrzik, és olyanok melyek az eltelt időt vizsgálják. Sok esetben elég lenne az is, ha illeszkedik a mintára, vagy egy értéknél nagyobb a visszakapott idő. Azonban ilyent KTest segítségével nem lehet megvalósítani.

\end{document}
